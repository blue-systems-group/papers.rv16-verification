\subsection{Searching Strategies}
\label{subsec:search}

In this section, we present an \textit{exhaustive} search algorithm of the
accepting transition sequence of $S^+$ on sniffer trace $Tr$. It is guaranteed to
yield an accepting sequence if there exists one, thus is exhaustive.  In the next
sections, we present heuristics to limit the search to accepting sequences of
$S^+$ that require relatively fewer transitions from $E^+_1 \cup E^+_2$.  Due to
the NP-completeness of the problem, this also makes the algorithm meaningful in
practice.

\begin{figure}[t!]
  \begin{minipage}{\textwidth}
    \begin{algorithm}[H]
      \caption{Exhaustive search algorithm of $S^+$ on $Tr$.}
      \label{alg:search}
      
      \begin{algorithmic}[1]
        \Function{search}{S, Tr}
          \Let{$S^+$}{\Call{augment}{S}}
          \State\Return{\Call{extend}{[], Tr, $S^+.s_0$}}
        \EndFunction
        \Function{extend}{prefix, p::suffix, $s$}
          \If{not \Call{likely}{prefix}}
            \label{alg:search:stop}
            {\Return{\nil}}\footnote{This check should be ignored in the exhaustive algorithm.}
          \EndIf
          \For{i $\in$ [0, 1, 2]}
          \label{alg:search:order}
          \Let{mutation}{EXTEND-i(prefix, p::suffix, $s$)}
            \If{mutation $\ne$ \nil}
              {\Return{mutation}}
            \EndIf
          \EndFor
          \State\Return{\nil}
        \EndFunction
        \Function{extend-0}{prefix, p::suffix, $s$}
          \For{$\langle s, s', p\rangle\footnote{$\langle s, s', p \rangle$ is
              short for $\langle s, *, s', *, p, *, *\rangle$} \in E$}
            \If{suffix = \nil}
              {\Return{prefix@p}}
              \label{alg:search:terminate_0}
            \EndIf
            \Let{mutation}{\Call{extend}{prefix@p, suffix, $s'$}}
            \label{alg:search:append}
            \If{mutation $\ne$ \nil}
              {\Return{mutation}}
            \EndIf
          \EndFor
          \State{\Return{\nil}}
        \EndFunction
        \Function{extend-1}{prefix, p::suffix, $s$}
          \ForAll{$\langle s, s', q\rangle \in E^+_1$}
            \If{q.time $>$ p.time}
              \label{alg:search:type_1_limit}
              \State{\textbf{continue}}
            \EndIf
            \Let{mutation}{\Call{extend}{prefix@q, p::suffix, $s'$}}
            \label{alg:search:guess}
            \If{mutation $\ne \nil$}
              {\Return{mutation}}
            \EndIf
          \EndFor
          \State{\Return{\nil}}
        \EndFunction
        \Function{extend-2}{prefix, p::suffix, $s$}
          \ForAll{$\langle s, s, p \rangle \in E^+_2$}
            \If{suffix = \nil}
              {\Return{prefix}}
              \label{alg:search:terminate_2}
            \EndIf
            \Let{mutation}{\Call{extend}{prefix, suffix, $s$}}
            \label{alg:search:drop}
            \If{mutation $\ne$ \nil}
              {\Return{mutation}}
            \EndIf
          \EndFor
          \State{\Return{\nil}}
        \EndFunction
      \end{algorithmic}
    \end{algorithm}
    \renewcommand\footnoterule{}
  \end{minipage}
  \vspace*{-5mm}
\end{figure}

The main routines of the algorithm are shown in Algorithm~\ref{alg:search}.  In
the top level \texttt{SEARCH} routine, we first obtain the augmented state
machine $S^+$, and then call the recursive \texttt{EXTEND} function with an empty
prefix, the sniffer trace, and the $S^+$'s initial state. 
In the \texttt{EXTEND} function, we try to consume the first packet in the
remaining trace using either Type-0, Type-1 or Type-2 transition.
Note that we always try to use Type-0 transitions before other two augmented
transitions (line~\ref{alg:search:order}).
This ensures the first found mutation trace will have the most number of Type-0
transitions among all possible mutation traces.
Intuitively, this means the search algorithm tries to utilize the sniffer's
observation as much as possible before being forced to make assumptions.


Each of the extend functions either returns the mutation trace $Tr'$, or
\textit{\nil} if the search fails.
In both \texttt{EXTEND-0} and
\texttt{EXTEND-2} function, if there is a valid transition, we try to consume
the next packet either by appending it to the prefix
(line~\ref{alg:search:append}) or dropping it (line~\ref{alg:search:drop}).
While in \texttt{EXTEND-1}, we guess a missing packet without consuming the next
real packet (line~\ref{alg:search:guess}).
Note that since only Type-0 and Type-2 consume packets, the recursion terminates
if there is a valid Type-0 or Type-2 transition for the last packet
(line~\ref{alg:search:terminate_0} and line~\ref{alg:search:terminate_2}).



It is not hard to see that Algorithm~\ref{alg:search} terminates on any sniffer
traces. Each node in the transition tree only has finite number of possible next
steps, and the depth of Type-1 transitions is limited by the time available
before the next packet (line~\ref{alg:search:type_1_limit}).
