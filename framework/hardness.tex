\subsection{Problem Hardness}
\label{subsec:hard}

In this section, we show that the VALIDATION problem is NP-complete. In
fact, the problem is still NP-complete even with only one type of augmented
transitions.

Recall that Type-1 transitions are added because the sniffer may miss packets.
Suppose an imaginary sniffer that is able to capture \textit{every} packet ever
transmitted, then only Type-2 transitions are needed since the sniffer may still
overhear packets sent to the DUT. Similarly, suppose another special sniffer
that would not overhear any packets sent to the DUT, then only Type-1
transitions are needed to infer missing packets.

We refer the augmented state machine that only has Type-0 and Type-1
transitions as $S^+_1$, and the augmented state machine that only has Type-0 and
Type-2 transitions as $S^+_2$. And we show that each subproblem of determining
trace satisfiability is NP-complete.

\begin{problem}
  VALIDATION-1\\
  Given that $Tr\setminus Tr_{DUT}=\emptyset$ (sniffer does not overhear
  packets).\\
  \textbf{instance} Checker state machine $S$ and sniffer trace $Tr$.\\
  \textbf{question} Does $S^+_1$ accept $Tr$?
\end{problem}

\begin{problem}
  VALIDATION-2\\
  Given that $Tr_{DUT} \subseteq Tr$ (sniffer does not missing packets).\\
  \textbf{instance} Checker state machine $S$ and sniffer trace $Tr$.\\
  \textbf{question} Does $S^+_2$ accept $Tr$?
\end{problem}

\begin{lemma}
  \label{lem:sub}
  Both VALIDATION-1 and VALIDATION-2 are NP-complete.
\end{lemma}
\begin{proof}
  First, note that the length of mutation trace $Tr'$ is polynomial to the
  length of $Tr$ because of the discrete time stamp and non-overlapping packets
  assumption.
  Therefore, given a state transition sequence as witness, it can be verified in
  polynomial time whether or not it is an accepting transition sequence, hence
  both VALIDATION-1 and VALIDATION-2 are in NP.

  Next, we show how the SAT problem can be reduced to either one of the two
  problems.
  Consider an instance of SAT problem of a propositional formula $F$ with $n$
  variables $x_0,x_1,\ldots, x_{n-1}$, we construct a corresponding protocol and
  its monitor state machine as follows.

  The protocol involves two devices: the DUT (transmitter) and the endpoint
  (receiver).
  The DUT shall send a series of packets, $pkt_0, pkt_1,\ldots, pkt_{n-1}$.
  For each $pkt_i$, if the DUT receives the
  acknowledgment packet $ack_i$ from the endpoint, it sets boolean variable
  $x_i$ to be \texttt{true}.
  Otherwise $x_i$ remains to be \texttt{false}.
  After $n$ rounds, the DUT evaluate the formula $F$ using the assignments and
  sends a special packet, $pkt_{true}$, if $F$ is \texttt{true}.
  One round of the protocol operation can be completed in polynomial time since
  any witness of $F$ can be evaluated in polynomial time.

  \begin{figure}[h!]
    \centering
    \includegraphics[width=0.7\textwidth]{./figures/sat_sm.pdf}
    \caption{\textbf{Monitor State Machine for SAT Problem.}}
    \label{fig:sat}
  \end{figure}



  The protocol monitor state machine $S$ is shown in Fig.~\ref{fig:sat}.
  Initially, all $x_i$ is set to \texttt{false}.
  At state $s_0$, the DUT shall transmit $pkt_i$ within a unit time, transit to
  $s_1$ and reset the clock along the transition.
  At state $s_1$, either the DUT receives the $ack_i$ packet and
  set $x_i$ to be \texttt{true} ($s_1 \rightarrow s_0$), or the DUT continues to
  transmit the next packet $pkt_{i+1}$.
  After $n$ rounds, the state machine is $s_0$ or $s_1$ depending on whether
  $ack_{n-1}$ is received by the DUT.
  In either case, the DUT shall evaluate $F$ and transmit $pkt_{true}$ if $F$ is
  \texttt{true}.  

  \sloppy{%
    Consider a sniffer trace $Tr_1=\{(0, pkt_0), (2, pkt_1), (4, pkt_2),\ldots,
  (2(n-1), pkt_{n-1}), (2n, pkt_{true})\}$.  That is, the sniffer only captures
  all $pkt_i$ plus the final $pkt_{true}$, but none of $ack_i$.  It is easy to see
  that $F$ is satisfiable if $S_1^+$ accepts $Tr_1$.  In particular, a successful
  run of $S_1^+$ on $Tr_1$ would have to guess, for each $pkt_i$, whether the
  Type-1 empty transitions should be used to infer the missing $ack_i$ packet,
  such that $F$ is \textit{true} at the end.  Note that for $Tr_1$, no Type-2 self
  transitions can be used since all packets in $Tr_1$ are sent from the DUT.
  Therefore, the SAT problem of $F$ can be reduced to the VALIDATION-1 problem of
  $S^+_1$ on sniffer trace $Tr_1$.
  }

  \sloppy{%
    On the other hand, consider another sniffer trace $Tr_2 =\{(0, pkt_0), (1,
    ack_0), (2, pkt_1), (3, ack_1),\ldots, (2n-2, pkt_{n-1}), (2n-1, ack_{n-1}), (2n,
  pkt_{true}\}$. That is, the sniffer captures all $n$ pair of $pkt_i$ and $ack_i$
  packets and the final $pkt_{true}$ packet.  Similar to $Tr_1$, $F$ is
  satisfiable if $S^+_2$ accepts $Tr_2$. A successful transition sequence of
  $S^+_2$ on $Tr_2$ must decide for each $ack_i$ packet, whether Type-2 self
  transitions should be used, so that $F$ can be evaluated as true at the end.
  Therefore, the SAT problem of $F$ can also be reduced to the VALIDATION-2
  problem of $S^+_2$ on sniffer trace $Tr_2$.
}

Since SAT is known to be NP-complete, both the VALIDATION-1 and the
VALIDATION-2 problem are also NP-complete.

\end{proof}

The hardness statement of the general VALIDATION problem naturally follows
Lemma~\ref{lem:sub}.

\begin{theorem}
  VALIDATION is NP-complete.
\end{theorem}
