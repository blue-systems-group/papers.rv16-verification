\subsection{Rate Control}

Rate control plays an important role in wireless performance. In this section,
we first briefly describe Minstrel---one of the most widely used rate control
algorithm. Then we show how we can apply the verification techniques on
Minstrel.

\subsubsection{Minstrel Rate Control Algorithm}

\sloppy{%
  Minstrel is the default rate control algorithm in Linux \texttt{mac80211}
  framework. The core of the algorithm is the multi-rate retry chain denoted as
  $\mathcal{C}=\langle r_1, c_1\rangle, \langle r_2, c_2\rangle,\ldots, \langle
  r_n, c_n\rangle$. The retry chain instruct the hardware to first try to
  transmit a packet for at most $c_1$ times on rate $r_1$, then at most $r_2$
  times on rate $r_2$ and so forth, until the packet is acknowledged. The
  hardware will abort transmitting the packet if it has been attempted for
  $\sum_{i=1}^{n}c_i$ times and no acknowledged is received.
}

In the Minstrel implementation of \texttt{mac80211}, the retry chain size $n$ is
4, and the rates in each retry stage are determined in Table~\ref{tab:retry}.
Note that once the rates are decided, the retry count $c_i$ is configured so
that the each retry stage does not exceed 26~ms to accommodate upper layer TCP
congestion control.


\begin{table}[h]
  \centering
  \caption{\textbf{Multi-Rate Retry Chain in Minstrel.} $\text{TP}_{r}$ is the
    estimated throughput of rate $r$.}
  \label{tab:retry}
  \begin{tabular}{cccc}
    \toprule
    \multirow{2}{*}{\textbf{Try}} & \multicolumn{2}{c}{\textbf{Probing Packet}} & \multirow{2}{*}{\textbf{Normal Packets}}\\ &
    $\text{TP}_{r_{\text{rnd}}} < \text{TP}_{r_{\text{tp}}}$
    & $\text{TP}_{r_{\text{rnd}}} > \text{TP}_{r_{\text{tp}}}$ & \\
    \midrule
    1 & $r_{\text{tp}}$ & $r_{\text{rnd}}$ & $r_{\text{tp}}$\\
    2 & $r_{\text{rnd}}$ & $r_{\text{tp}}$ & $r_{\text{tp2}}$\\
    3 & $r_{\text{prob}}$ & $r_{\text{prob}}$ & $r_{\text{prob}}$\\
    4 & $r_{\text{base}}$ & $r_{\text{base}}$ & $r_{\text{base}}$\\
    \bottomrule
  \end{tabular}
\end{table}

For 90\% packets (normal), the rate with best estimated throughput
($r_{\text{tp}}$) is used first, followed by the rate with 2nd best estimated
throughput ($r_{\text{tp2}}$), then the rate with best success probability
($r_{\text{prob}}$), and finally the lowest base rate ($r_{\text{base}}$). For
the rest 10\% packets (probing), Minstrel randomly probes other rates, and the
retry chain for probing packets are determined so that the rate with higher
throughput will be used first.

Minstrel collects the success statistics of packets transmitted during last
interval, and updates the estimated throughput for each rate as follows:
\begin{align}
  \text{TP}_{r} &= r \times \text{Prob}_r\\
  \text{Prob}_{r} &= (1-\alpha)\times\text{Prob}^*_{r} + \alpha\times\text{Prob}_r\\
  \text{Prob}^*_r &= \frac{\text{Success Packets \#}}{\text{Total Packets \#}}
\end{align}

Where $\text{Prob}^*_{r}$ is the packet success probability during last
interval, and $\text{Prob}_{r}$ is the Exponential Weighted Moving Average
(EWMA) of $\text{Prob}^*_{r}$. The default value of the EWMA weight $\alpha$ is
0.75.

\subsection{Testing Minstrel Implementation}
