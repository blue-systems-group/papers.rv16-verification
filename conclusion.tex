\section{Conclusions and Discussions}
\label{sec:conclusion}

In this paper, we formally describe the uncertainty problem in validating
wireless protocols using sniffers. We propose to systematically augment the
protocol state machine to explicitly encode the inherent uncertainty of sniffer
traces. We characterize the NP-completeness of the problem and propose practical
heuristics to restrict the search to only traces with high likelihood.  We
implement and evaluate our framework using both \ns{} simulator and real world
traces, and show that our framework can tolerance sniffer trace uncertainties
and report true violations. Finally, we discuss a few challenges and future
directions.

\textbf{Verification Coverage.} Given a single sniffer trace, it is possible
that not all the states in the state machine are visited during the verification
process. For instance, a rate control state machine based on certain consecutive
packet losses patterns can not be verified if no such consecutive losses appear
in the sniffer trace. In general, given a protocol state machine, how to extract
the packet patterns for each state to be reached and how to alter the testing
such that such patterns can be observed?

\textbf{State Machine Generation.} We manually translated the protocols studied
in this paper into checker state machines based on the source code, comments and
documentation. The process is time-consuming and error-prone. A more scalable
approach would be taking the protocol specification written in certain formal
language, and automatically translate such specification into state machines
that can be used for verification process.

