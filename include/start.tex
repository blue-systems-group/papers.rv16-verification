\documentclass{llncs}
\usepackage{booktabs}
\usepackage{subcaption}
\usepackage{comment}
\usepackage{mathrsfs}
\usepackage{balance}
\usepackage{enumitem}
\usepackage{hyperref}
\usepackage[noend]{algpseudocode}
\usepackage[group-separator={,}]{siunitx}
\usepackage{caption}
\usepackage{mathtools}
\usepackage{amssymb}
\usepackage{amsfonts}
\usepackage{listings}
\usepackage{algorithm}
\usepackage{algorithmicx}
\usepackage{bm}
\usepackage[all]{hypcap}
\usepackage[absolute]{textpos}
\usepackage{wrapfig}
\usepackage[title]{appendix}

%\emergencystretch=10pt

\renewcommand{\algorithmicrequire}{\textbf{Input:}}
\renewcommand{\algorithmicensure}{\textbf{Output:}}
\algnewcommand{\LineComment}[1]{\State{\(\triangleright\) #1}}
\newcommand*\Let[2]{\State #1 := #2}


\hypersetup{%
  bookmarks=true,
  unicode=true,
  pdftoolbar=true,
  pdfmenubar=true,
  pdffitwindow=true,
  pdfstartview={FitV},
  pdfnewwindow=true,
  colorlinks=false,
  pdfdisplaydoctitle=true,
  pdfborder={0 0 0}
}

\setlength{\TPHorizModule}{1in}
\setlength{\TPVertModule}{1in}
\textblockorigin{0.75in}{0.875in}

\setlist[itemize]{leftmargin=*,partopsep=5pt}
\setlist[enumerate]{leftmargin=*,partopsep=5pt}

% \newcommand*{\refname}{References}
\paperheight 11in
\paperwidth 8.5in

\usepackage{dcolumn}
\newcolumntype{d}[1]{D{.}{.}{#1}}

\newcommand\nil{\textit{nil}}

\usepackage{tikz}
\newcommand*\circled[1]{\tikz[baseline=(char.base)]{
      \node[shape=circle,draw,inner sep=0pt] (char) {#1};}}

% \newdef{definition}{Definition}
%\newtheorem{theorem}{Theorem}
%\newtheorem{corollary}{Corollary}
%\newtheorem{problem}{Problem}
%\newtheorem{lemma}{Lemma}

%\numberwithin{theorem}{section}
%\numberwithin{corollary}{section}
%\numberwithin{problem}{section}
%\numberwithin{lemma}{section}

\makeatletter
\newcommand*{\rome}[1]{\expandafter\@slowromancap\romannumeral #1@}
\makeatother

% 16 Nov 2010 : GWA : Any special macros or other stuff for this particular
%               paper go here.

\newcommand{\wifi}{Wi-Fi}
\newcommand{\codename}{\textsc{Verifier}}
\newcommand{\us}{$\mu$s}
\newcommand{\ns}{\textsc{NS-3}}

\newcommand{\shuvendu}[1]{{\bf [[SKL:{#1}]]}}

