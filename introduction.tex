% This fixes the balance on the first page.

\section{Introduction}
\label{sec:intro}

Custom wireless protocols are being designed and deployed to meet the specific
performance and power needs of special-purpose wireless devices such as
wearables~\cite{iris}, game controllers, and mobile computing devices.
%
Validating that these devices correctly implement the protocol is crucial to
achieve the design goals of the protocol and also prevent bugs in shipping
products (e.g., Apple AWDL protocol bug in iOS 8~\cite{wifried}, \wifi{} issues
in Android Lollipop~\cite{lollipop} and Microsoft Surface 3~\cite{surface}).


But validating the protocol implementation on such devices is challenging
because collecting traces from the device under test (DUT) is often
infeasible.
%
The resource limitations of embedded or battery-powered devices may cause them
to not provide trace collecting capabilities.
%
Devices may include proprietary hardware or firmware that hides the protocol
details.
%
This may occur when multiple devices from different vendors are being tested
for interoperability, or when development has been performed by a third-party
that considers the implementation proprietary.
%
And even if direct instrumentation is possible, the overhead it causes may alter
the behavior of the DUT~\cite{mytkowicz2008observer}, threatening the validation
results.

An attractive alternative is to use wireless
sniffers to record traffic generated by the DUTs during testing.
%
Sniffers do not require direct access to the DUT or alter its behavior.
%
However, due to the fundamentally unpredictable nature of wireless
communications, the packets captured by the sniffer will not exactly match
those received by the DUT.
%
The sniffer may miss packets that the DUT received, or receive packets that
the DUT missed.
%
This is true even when using multiple
sniffers~\cite{cheng2006jigsaw,mahajan2006analyzing,bahl2006enhancing}, sniffer
with multiple antennas~\cite{omnipeek}, or in isolated wireless environments.
%
(Isolated environments are also inappropriate for testing in the common case
when the DUT must cope with interference encountered in uncontrolled wireless
environments.)

Because the sniffer trace does not perfectly match the actual trace,
uncertainty arises during protocol implementation validation.
%
For example, if the DUT fails to respond correctly to a packet in the sniffer
trace, it may be because (a) the DUT's implementation is incorrect, (b) the DUT
did not actually receive the packet or (c) the DUT's response was missed by the
sniffer.
%
Whenever the DUT's behavior does not match the specification, there are now two
potential explanations:
%
either the DUT's implementation is wrong, or the sniffer trace is incorrect.
%
Accurate validation requires accurately distinguishing between these two
causes.

We present a new technique enabling accurate validation of protocol
implementation using wireless sniffers.
%
Given a state machine representing the protocol being validated, we describe a
systematic transformation that adds nondeterministic transitions to
incorporate uncertainty introduced by the sniffer.
%
This augmented validation state machine processes the sniffer trace into a set
of mutated traces, each satisfying the original state machine with certain
probability.
%
If the set is empty, the implementation definitely violates the protocol.
%
If the set contains only low-probability traces, then the implementation
probably violates the protocol.
%
Finding all mutated traces is NP-complete, but the approach can be made
practical by applying protocol-oblivious heuristics that limit the search to
likely mutated traces.

Our paper makes the following contributions:
%
\begin{itemize}
		%
  \item To the best of our knowledge we are the first to identify the
    uncertainty problem caused by sniffer in validating wireless protocol
    implementations.
		%
  \item We formalize the problem using a nondeterministic state machine that
    systematically and completely encodes the inherent uncertainty of the
    sniffer trace.
		%
    \item We characterize the NP-completeness of the validating problem, and
      present protocol-oblivious heuristics to prune the search
      space and make validation possible in practice.
		%
	\item We implement the validation framework and evaluate it using
    \ns{} network simulator~\cite{riley2010ns}.
		%
    Our framework accurately identifies both introduced as well as previously
    unknown violations in \ns{}'s implementations of the 802.11 protocol.
		%
\end{itemize}

The rest of this paper is organized as follows.
%
We motivate the uncertainty problem in Section~\ref{sec:model}.
%
We then formally describe the problem in Section~\ref{sec:framework},
including the completeness of the augmentation (\S~\ref{subsec:augment}),
hardness analysis (\S~\ref{subsec:hard}) and search algorithms
(\S~\ref{subsec:search}).
%
We continue by evaluating our framework through two case studies in
Section~\ref{sec:case}.  Finally, we present related works in
Section~\ref{sec:related} and concludes in Section~\ref{sec:conclusion}.
